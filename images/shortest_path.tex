% Example for illustrating variational methods
% Jeremy Roberts, 2013
\documentclass{article}

\usepackage{gnuplot-lua-tikz}
\usepackage[active,tightpage]{preview} 
\PreviewEnvironment{tikzpicture}
\setlength\PreviewBorder{2mm}

\begin{document}
%\pagestyle{empty}

\begin{tikzpicture}[gnuplot]
  % X and Y axes
  \draw[-stealth] (0.0, 0.0) -- (8.0, 0.0) node[right] {$x$};
  \draw[-stealth] (0.0, 0.0) -- (0.0, 6.0) node[above] {$y$};
  % Two points
  \fill(1.0, 1.0) circle (2pt) node[below] {{\small $(x_1,\, y_1)$}}; 
  \fill(7.0, 5.0) circle (2pt) node[above] {{\small $(x_2,\, y_2)$}};
  % Actual line connecting
  \draw[color=black] plot[id=y, domain=1.0:7.0] function{(4.0/6.0)*x+1.0/3.0};
  \node[] at (4.0, 3.2) {$f$};
  % Two potential lines
  \draw[color=black,densely dashed] plot[id=g1, domain=1.0:7.0] 
    function{(4.0/6.0)*x+1.0/3.0 - 0.05*(sin(1*x)**2+1)*(x-1.0)*(x-7.0)};
  \node[] at (4.7, 4.7) {$f + \epsilon g_1$};
  \draw[color=black,loosely dashed] plot[id=g2, domain=1.0:7.0] 
    function{(4.0/6.0)*x+1.0/3.0 + 0.05*(sin(1.5*x)**2+1)*(x-1.0)*(x-7.0)};
  \node[] at (3.2, 1.3) {$f+\epsilon g_2$};
\end{tikzpicture}

\end{document}
