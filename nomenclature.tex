\chapter{Nomenclature}

To facilitate understanding of the various terms used throughout the lecture materials, we provide here a list of variables, short definitions, and common units where applicable.  In very few cases, symbols are used more than once due to convention (e.g. $\phi$ for both flux and azimuthal angle).  Bold symbols indicate a vector quantity.

\begin{table}[th]
 \caption{Fundamental quantities}
 \begin{center} 
 {\small
 \begin{tabular*}{0.90\textwidth}{@{\extracolsep{\fill}} ccc } 
  \toprule 
   Symbol                            & Description   & Units  \\
  \midrule 
   $\psi(\vec{r},\hat{\Omega},E,t)$          & angular flux                  & $\frac{\mathrm{n}}{\mathrm{cm^2-s-eV-ster}}$  \\
   $\psi^+(\vec{r},\hat{\Omega},E,t)$        & adjoint angular flux          & $\frac{\mathrm{n}}{\mathrm{cm^2-s-eV-ster}}$  \\
   $\phi(\vec{r},E,t)$                       & scalar  flux                  & $\frac{\mathrm{n}}{\mathrm{cm^2-s-eV}}$  \\
   $\phi^+(\vec{r},E,t)$                     & adjoint scalar flux           & $\frac{\mathrm{n}}{\mathrm{cm^2-s-eV}}$  \\
   $\mathbf{j}(\vec{r},\hat{\Omega},E,t)$    & angular current density       & $\frac{\mathrm{n}}{\mathrm{cm^2-s-eV-ster}}$  \\
   $\mathbf{ J}(\vec{r},E,t) $               & current density               & $\frac{\mathrm{n}}{\mathrm{cm^2-s-eV}}$  \\
   $J_{\pm}(\vec{r},E,t)$                    & partial current density       & $\frac{\mathrm{n}}{\mathrm{cm^2-s-eV}}$  \\
  \bottomrule 
 \end{tabular*} 
 }
 \end{center} 
 \label{tbl:snmeshstudy}  
\end{table}

